% Options for packages loaded elsewhere
\PassOptionsToPackage{unicode}{hyperref}
\PassOptionsToPackage{hyphens}{url}
%
\documentclass[
]{article}
\usepackage{amsmath,amssymb}
\usepackage{lmodern}
\usepackage{ifxetex,ifluatex}
\ifnum 0\ifxetex 1\fi\ifluatex 1\fi=0 % if pdftex
  \usepackage[T1]{fontenc}
  \usepackage[utf8]{inputenc}
  \usepackage{textcomp} % provide euro and other symbols
\else % if luatex or xetex
  \usepackage{unicode-math}
  \defaultfontfeatures{Scale=MatchLowercase}
  \defaultfontfeatures[\rmfamily]{Ligatures=TeX,Scale=1}
\fi
% Use upquote if available, for straight quotes in verbatim environments
\IfFileExists{upquote.sty}{\usepackage{upquote}}{}
\IfFileExists{microtype.sty}{% use microtype if available
  \usepackage[]{microtype}
  \UseMicrotypeSet[protrusion]{basicmath} % disable protrusion for tt fonts
}{}
\makeatletter
\@ifundefined{KOMAClassName}{% if non-KOMA class
  \IfFileExists{parskip.sty}{%
    \usepackage{parskip}
  }{% else
    \setlength{\parindent}{0pt}
    \setlength{\parskip}{6pt plus 2pt minus 1pt}}
}{% if KOMA class
  \KOMAoptions{parskip=half}}
\makeatother
\usepackage{xcolor}
\IfFileExists{xurl.sty}{\usepackage{xurl}}{} % add URL line breaks if available
\IfFileExists{bookmark.sty}{\usepackage{bookmark}}{\usepackage{hyperref}}
\hypersetup{
  pdftitle={Assignments},
  hidelinks,
  pdfcreator={LaTeX via pandoc}}
\urlstyle{same} % disable monospaced font for URLs
\usepackage[margin=1in]{geometry}
\usepackage{color}
\usepackage{fancyvrb}
\newcommand{\VerbBar}{|}
\newcommand{\VERB}{\Verb[commandchars=\\\{\}]}
\DefineVerbatimEnvironment{Highlighting}{Verbatim}{commandchars=\\\{\}}
% Add ',fontsize=\small' for more characters per line
\usepackage{framed}
\definecolor{shadecolor}{RGB}{248,248,248}
\newenvironment{Shaded}{\begin{snugshade}}{\end{snugshade}}
\newcommand{\AlertTok}[1]{\textcolor[rgb]{0.94,0.16,0.16}{#1}}
\newcommand{\AnnotationTok}[1]{\textcolor[rgb]{0.56,0.35,0.01}{\textbf{\textit{#1}}}}
\newcommand{\AttributeTok}[1]{\textcolor[rgb]{0.77,0.63,0.00}{#1}}
\newcommand{\BaseNTok}[1]{\textcolor[rgb]{0.00,0.00,0.81}{#1}}
\newcommand{\BuiltInTok}[1]{#1}
\newcommand{\CharTok}[1]{\textcolor[rgb]{0.31,0.60,0.02}{#1}}
\newcommand{\CommentTok}[1]{\textcolor[rgb]{0.56,0.35,0.01}{\textit{#1}}}
\newcommand{\CommentVarTok}[1]{\textcolor[rgb]{0.56,0.35,0.01}{\textbf{\textit{#1}}}}
\newcommand{\ConstantTok}[1]{\textcolor[rgb]{0.00,0.00,0.00}{#1}}
\newcommand{\ControlFlowTok}[1]{\textcolor[rgb]{0.13,0.29,0.53}{\textbf{#1}}}
\newcommand{\DataTypeTok}[1]{\textcolor[rgb]{0.13,0.29,0.53}{#1}}
\newcommand{\DecValTok}[1]{\textcolor[rgb]{0.00,0.00,0.81}{#1}}
\newcommand{\DocumentationTok}[1]{\textcolor[rgb]{0.56,0.35,0.01}{\textbf{\textit{#1}}}}
\newcommand{\ErrorTok}[1]{\textcolor[rgb]{0.64,0.00,0.00}{\textbf{#1}}}
\newcommand{\ExtensionTok}[1]{#1}
\newcommand{\FloatTok}[1]{\textcolor[rgb]{0.00,0.00,0.81}{#1}}
\newcommand{\FunctionTok}[1]{\textcolor[rgb]{0.00,0.00,0.00}{#1}}
\newcommand{\ImportTok}[1]{#1}
\newcommand{\InformationTok}[1]{\textcolor[rgb]{0.56,0.35,0.01}{\textbf{\textit{#1}}}}
\newcommand{\KeywordTok}[1]{\textcolor[rgb]{0.13,0.29,0.53}{\textbf{#1}}}
\newcommand{\NormalTok}[1]{#1}
\newcommand{\OperatorTok}[1]{\textcolor[rgb]{0.81,0.36,0.00}{\textbf{#1}}}
\newcommand{\OtherTok}[1]{\textcolor[rgb]{0.56,0.35,0.01}{#1}}
\newcommand{\PreprocessorTok}[1]{\textcolor[rgb]{0.56,0.35,0.01}{\textit{#1}}}
\newcommand{\RegionMarkerTok}[1]{#1}
\newcommand{\SpecialCharTok}[1]{\textcolor[rgb]{0.00,0.00,0.00}{#1}}
\newcommand{\SpecialStringTok}[1]{\textcolor[rgb]{0.31,0.60,0.02}{#1}}
\newcommand{\StringTok}[1]{\textcolor[rgb]{0.31,0.60,0.02}{#1}}
\newcommand{\VariableTok}[1]{\textcolor[rgb]{0.00,0.00,0.00}{#1}}
\newcommand{\VerbatimStringTok}[1]{\textcolor[rgb]{0.31,0.60,0.02}{#1}}
\newcommand{\WarningTok}[1]{\textcolor[rgb]{0.56,0.35,0.01}{\textbf{\textit{#1}}}}
\usepackage{graphicx}
\makeatletter
\def\maxwidth{\ifdim\Gin@nat@width>\linewidth\linewidth\else\Gin@nat@width\fi}
\def\maxheight{\ifdim\Gin@nat@height>\textheight\textheight\else\Gin@nat@height\fi}
\makeatother
% Scale images if necessary, so that they will not overflow the page
% margins by default, and it is still possible to overwrite the defaults
% using explicit options in \includegraphics[width, height, ...]{}
\setkeys{Gin}{width=\maxwidth,height=\maxheight,keepaspectratio}
% Set default figure placement to htbp
\makeatletter
\def\fps@figure{htbp}
\makeatother
\setlength{\emergencystretch}{3em} % prevent overfull lines
\providecommand{\tightlist}{%
  \setlength{\itemsep}{0pt}\setlength{\parskip}{0pt}}
\setcounter{secnumdepth}{-\maxdimen} % remove section numbering
\ifluatex
  \usepackage{selnolig}  % disable illegal ligatures
\fi

\title{Assignments}
\author{}
\date{\vspace{-2.5em}}

\begin{document}
\maketitle

This page will contain all the assignments you submit for the class.

\hypertarget{assignment-1}{%
\section{Assignment 1}\label{assignment-1}}

\textbf{Collaborators: Theodora Athanitis }

\hypertarget{problem-1}{%
\subsubsection{Problem 1}\label{problem-1}}

Install the datasets package on the console below using
\texttt{install.packages("datasets")}. Now load the library.

\begin{Shaded}
\begin{Highlighting}[]
\CommentTok{\#install.packages("datasets")}
\end{Highlighting}
\end{Shaded}

\emph{Answer: Installed!}

Load the USArrests dataset and rename it \texttt{dat}. Note that this
dataset comes with R, in the package datasets, so there's no need to
load data from your computer. Why is it useful to rename the dataset?

\begin{Shaded}
\begin{Highlighting}[]
\NormalTok{dat}\OtherTok{\textless{}{-}}\NormalTok{USArrests}
\end{Highlighting}
\end{Shaded}

Answer: Renaming data sets ensures that you do not edit any of your
original data on accident, and it is also easier to type.

\hypertarget{problem-2}{%
\subsubsection{Problem 2}\label{problem-2}}

Use this command to make the state names into a new variable called
State.

\begin{Shaded}
\begin{Highlighting}[]
\NormalTok{dat}\SpecialCharTok{$}\NormalTok{state}\OtherTok{\textless{}{-}}\FunctionTok{tolower}\NormalTok{(}\FunctionTok{rownames}\NormalTok{(USArrests))}
\end{Highlighting}
\end{Shaded}

This dataset has the state names as row names, so we just want to make
them into a new variable. We also make them all lower case, because that
will help us draw a map later - the map function requires the states to
be lower case.

List the variables contained in the dataset \texttt{USArrests}.

\begin{Shaded}
\begin{Highlighting}[]
\FunctionTok{names}\NormalTok{(dat)}
\end{Highlighting}
\end{Shaded}

\begin{verbatim}
## [1] "Murder"   "Assault"  "UrbanPop" "Rape"
\end{verbatim}

\emph{Answer: The variables are Murder, Assault, UrbanPop, and Rape, in
addition to the new variable that we just created, state.}

\hypertarget{problem-3}{%
\subsubsection{Problem 3}\label{problem-3}}

What type of variable (from the DVB chapter) is \texttt{Murder}?

\emph{Answer: Murder is a quantitative variable according to the DVB
chapter because we are measuring the amount of murder arrests per
100,000.}

What R Type of variable is it?

\emph{Answer: In R, the data points in the Murder column are numeric
variables. However, the object itself is identified as a character, as
seen below}

\begin{Shaded}
\begin{Highlighting}[]
\FunctionTok{typeof}\NormalTok{(}\StringTok{"Murder"}\NormalTok{)}
\end{Highlighting}
\end{Shaded}

\begin{verbatim}
## [1] "character"
\end{verbatim}

\hypertarget{problem-4}{%
\subsubsection{Problem 4}\label{problem-4}}

What information is contained in this dataset, in general? What do the
numbers mean?

\begin{Shaded}
\begin{Highlighting}[]
\FunctionTok{head}\NormalTok{(dat)}
\end{Highlighting}
\end{Shaded}

\begin{verbatim}
##            Murder Assault UrbanPop Rape
## Alabama      13.2     236       58 21.2
## Alaska       10.0     263       48 44.5
## Arizona       8.1     294       80 31.0
## Arkansas      8.8     190       50 19.5
## California    9.0     276       91 40.6
## Colorado      7.9     204       78 38.7
\end{verbatim}

\emph{Answer: Generally, this dataset contains rates for specific crime
arrests per 100,000 people, as well as percentages of urban population
per state. The numbers are equivalent to the calculated arrest rates for
Murder, Assault, and Rape. The Urban Population numbers are the
percentages of the population in each state that live in an urban area.}

\hypertarget{problem-5}{%
\subsubsection{Problem 5}\label{problem-5}}

Draw a histogram of \texttt{Murder} with proper labels and title.

\begin{Shaded}
\begin{Highlighting}[]
\FunctionTok{hist}\NormalTok{(dat}\SpecialCharTok{$}\NormalTok{Murder, }\AttributeTok{xlab =} \StringTok{"Murder Arrests per 100,000 People"}\NormalTok{, }\AttributeTok{main =} \StringTok{"Frequency of Murder Arrest Rates in the US"}\NormalTok{, }\AttributeTok{col =} \StringTok{"purple"}\NormalTok{)}
\end{Highlighting}
\end{Shaded}

\includegraphics{Assignments_files/figure-latex/unnamed-chunk-7-1.pdf}

\hypertarget{problem-6}{%
\subsubsection{Problem 6}\label{problem-6}}

Please summarize \texttt{Murder} quantitatively. What are its mean and
median? What is the difference between mean and median? What is a
quartile, and why do you think R gives you the 1st Qu. and 3rd Qu.?

\hypertarget{problem-7}{%
\subsubsection{Problem 7}\label{problem-7}}

Repeat the same steps you followed for \texttt{Murder}, for the
variables \texttt{Assault} and \texttt{Rape}. Now plot all three
histograms together. You can do this by using the command
\texttt{par(mfrow=c(3,1))} and then plotting each of the three.

What does the command par do, in your own words (you can look this up by
asking R \texttt{?par})?

Answer:

What can you learn from plotting the histograms together?

Answer:

\hypertarget{problem-8}{%
\subsubsection{Problem 8}\label{problem-8}}

In the console below (not in text), type
\texttt{install.packages("maps")} and press Enter, and then type
\texttt{install.packages("ggplot2")} and press Enter. This will install
the packages so you can load the libraries.

Run this code:

\begin{Shaded}
\begin{Highlighting}[]
\FunctionTok{library}\NormalTok{(}\StringTok{\textquotesingle{}maps\textquotesingle{}}\NormalTok{) }
\FunctionTok{library}\NormalTok{(}\StringTok{\textquotesingle{}ggplot2\textquotesingle{}}\NormalTok{) }

\FunctionTok{ggplot}\NormalTok{(dat, }\FunctionTok{aes}\NormalTok{(}\AttributeTok{map\_id=}\NormalTok{state, }\AttributeTok{fill=}\NormalTok{Murder)) }\SpecialCharTok{+} 
  \FunctionTok{geom\_map}\NormalTok{(}\AttributeTok{map=}\FunctionTok{map\_data}\NormalTok{(}\StringTok{"state"}\NormalTok{)) }\SpecialCharTok{+} 
  \FunctionTok{expand\_limits}\NormalTok{(}\AttributeTok{x=}\FunctionTok{map\_data}\NormalTok{(}\StringTok{"state"}\NormalTok{)}\SpecialCharTok{$}\NormalTok{long, }\AttributeTok{y=}\FunctionTok{map\_data}\NormalTok{(}\StringTok{"state"}\NormalTok{)}\SpecialCharTok{$}\NormalTok{lat)}
\end{Highlighting}
\end{Shaded}

What does this code do? Explain what each line is doing.

Answer:

\[\\[2in]\]

\end{document}
